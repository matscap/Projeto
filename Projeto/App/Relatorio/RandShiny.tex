\textbf{\R}  it's a programming language created in 1993 by Ross Ihaka and Robert Gentleman and as we can see in the \cite{R-project website} R-project website,
\begin{quote}
    \textit{"\R is a language and environment for statistical computing and graphics."} 
\end{quote} 

\begin{figure}
\centering
\begin{subfigure}{.5\textwidth}
  \centering
  \includegraphics[width=3cm]{images/R.jpg}
  \label{fig:sub1}
\end{subfigure}%
\begin{subfigure}{.5\textwidth}
  \centering
  \includegraphics[width=5cm]{images/Rstudio.png}
  \label{fig:sub2}
\end{subfigure}
\caption{Logos of R and RStudio}
\label{fig:test}
\end{figure}


For our project, the main strength that \R provides us, is the
ability of produce good and well-designed plots and graphics to study the data that we have. The \ac{IDE} that we will use is \textbf{RStudio}.\\

\textbf{\Shiny} from RStudio is \cite{shiny} an \R package that makes it easy to build interactive web applications straight from \R. 
With this package we can combine the computational power of \R with the versatility and interactivity of modern web. This feature it's relevant to develop plots and graphics that can be understood by everyone, and looking pretty while doing it.