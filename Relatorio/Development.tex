The R Shiny framework is an RStudio package that makes it incredibly easy to build interactive web applications with R. A Shiny has the benefit of allowing us to create highly effective reports and data visualizations where the user can explore a set of data.

Shiny apps are divided into two parts: 

\begin{itemize}
    \item server.r
    \item ui.r (UI)
    
\end{itemize}
 % se tiver a der merd* corrijam 
 
The ui specification defines:
\begin{itemize}
    \item  Layout functions to configure the visual structure of the html page that will be generated when the application is executed. The fluidPage() function embeds and configures all the necessary and sufficient HTML, CSS and JavaScript code for the application. To create more complex layouts, you need to call layout functions inside fluidPage().
    \item Input control functions that will allow the user to interact with the application. Functions such as sliderInput(), selectInput(), textInput(), numericInput() can be used. All these functions have inputID as their first argument, simple string, with the same restrictions as the object names of R, and unique. This identifier is used to bind the ui to the server.
    If in the ui the inputID = “name”, the server will access it using input?name. The second argument, label, is also important as it contains the text that appears in the application's layout.
    \item Output controls that indicate where to place output with reactive behavior. Examples of output control functions are textOutput(), tableOutput(), among others. As with input control functions, the first argument must be the unique ID. If, for example, in the ui an ID with the name “plot” is defined, on the server the access is output$plot.



 In the server specification, functions that allow calculations to be performed and updated (using reactivity) are implemented. The server controls what data will be through the UI. The server will be where you upload and collate the data and then set your options (ie graphics) using input from the UI.
 
 
 For this purpose, specific rendering functions (render functions) are used. Each render{Type} function produces a specific type of output (for example, renderTable() produces tables, renderText() produces text). Usually these functions are paired with {Type}Output functions (for example renderTable() because it is paired with tableOutput()).


 
 Even though we created two separate files for our application, namely server.r and ui.r , shiny supports single file applications. A single file configuration puts both the server and user interface code in a single app.R file, whereas the multiple file configuration puts them in their own separate files. Functionally, these configurations will produce the same app. The multiple file configuration is generally preferred, especially for larger applications, as it usually makes code easier to manage. For smaller apps, the single file configuration is likely a more efficient way to go.
