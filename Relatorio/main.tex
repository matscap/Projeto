\documentclass[11pt,a4paper]{report}

\usepackage[english]{babel} 
\usepackage[utf8]{inputenc} % Unicode
\usepackage{graphicx} 
\usepackage{url} 
\usepackage{amsmath}
\usepackage{textcomp} 
\usepackage{multirow} 
\usepackage[pdftex,hidelinks]{hyperref}
\usepackage{listings}
\usepackage{natbib}
\usepackage{graphicx}
\usepackage{hyperref}
\usepackage{verbatim}
\usepackage{acronym}
\usepackage{ textcomp }

\def\R{{\textsl{R }}}
\def\Shiny{\textsl{Shiny }}
\def\sar{\textsl{SARS-CoV-2 }}



\title{Development of a \Shiny application to visualize SARS-CoV-2 vaccination data of Portugal}

\author{
  Bruna Araújo\\
  \texttt{A84408}
  \and
  Matias Capitão\\
  \texttt{A82726} \\ 
  \\ Mentoring:\\
  Stora Cecília \\
  \and
  Rafael Antunes\\
  \texttt{A77457}
}
\date{May or June 2021}



\begin{document}

\maketitle

\begin{abstract}
% Apagar daqui
    ************* Ao fim retornamos aqui.\\ Entretanto encontrei isto para nos ajudar a fazer o Abstract *************\\ \\
%até aqui
    Write the abstract at the very end, when you’ve completed the rest of the text. There are four things you need to include:\\ \\

    Your research problem and objectives\\
    Your methods\\
    Your key results or arguments\\
    Your conclusion

\end{abstract}

\tableofcontents
\newpage

\listoffigures
\newpage

\listoftables
\newpage

{\huge\textbf{Acronym List}}\\
\begin{acronym}[MPC] % Give the longest label here so that the list is nicely aligned
\acro{IDE}{Integrated Development Environment}
\acro{WHO}{World Healh Organization}
\acro{EU}{European Union}
\acro{ECDC}{European Centre for Disease Prevention and Control}

\end{acronym}


\chapter{Introduction}
\section{\R and \Shiny}

\textbf{\R}  it's a programming language created in 1993 by Ross Ihaka and Robert Gentleman and as we can see in the \cite{R-project website} R-project website,
\begin{quote}
    \textit{"\R is a language and environment for statistical computing and graphics."} 
\end{quote} 

\begin{figure}
\centering
\begin{subfigure}{.5\textwidth}
  \centering
  \includegraphics[width=3cm]{images/R.jpg}
  \label{fig:sub1}
\end{subfigure}%
\begin{subfigure}{.5\textwidth}
  \centering
  \includegraphics[width=5cm]{images/Rstudio.png}
  \label{fig:sub2}
\end{subfigure}
\caption{Logos of R and RStudio}
\label{fig:test}
\end{figure}


For our project, the main strength that \R provides us, is the
ability of produce good and well-designed plots and graphics to study the data that we have. The \ac{IDE} that we will use is \textbf{RStudio}.\\

\textbf{\Shiny} from RStudio is \cite{shiny} an \R package that makes it easy to build interactive web applications straight from \R. 
With this package we can combine the computational power of \R with the versatility and interactivity of modern web. This feature it's relevant to develop plots and graphics that can be understood by everyone, and looking pretty while doing it.

% Apagar aqui
\\
\\
\textbf{NÃO GOSTO DESTA ULTIMA FRASE. DESPOIS VER ESTA CENA.}

%Até aqui

\section{SARS-CoV-2}
\subsection{Brief description}

As we can read in \cite{covwiki},
"The COVID-19 pandemic, also known as the coronavirus pandemic, is an ongoing global pandemic of coronavirus disease 2019 (COVID-19), which is caused by severe acute respiratory syndrome coronavirus 2 (SARS-CoV-2). The virus was first identified in December 2019 in Wuhan, China. The World Health Organization declared a Public Health Emergency of International Concern on 30 January 2020, and later declared a pandemic on 11 March 2020. As of 27 June 2021, more than 180 million cases have been confirmed, with more than 3.91 million confirmed deaths attributed to COVID-19, making it one of the deadliest pandemics in history."
\\
The main symptoms of COVID-19 are fever, cough, breathing difficulties, loss of smell and taste. 
\\
It's a disease that attacks people of all age and is transmitted mainly by air, via particles that we expel. \\
There are several reasons for the pandemic to spread globally, mainly the fact that a patient may be asymptomatic, which can cause him to transmit the virus without knowing it. For the same reason, the fact that the incubation period is about 14 days is also a factor to be taken into account. 

\subsection{Vaccination}
Vaccines are substances made up of pathogens (viruses or bacteria), living or dead, or their derivatives. They stimulate the immune system to produce antibodies that act against pathogens that cause infections.

Vaccination is a way to protect people from harmful diseases. It uses the body's natural defenses to build resistance to specific infections and makes the
 stronger immune system.
It reduces the possibility of contracting the disease and thus prevents it from spreading.
\\
As soon as the COVID-19 pandemic took hold in the world, research for the production of safe and effective vaccines began immediately. Vaccination is critical to ending the COVID-19 pandemic. WHO is working tirelessly with partners to develop, manufacture and distribute safe and effective vaccines.
\\
Until May 10, 13 vaccines have been authorized by at least one national regulatory authority for public use. However, in Europe, after consulting the list of authorized vaccines on the official website of the \ac{EU} \cite {EU}, we have four vaccines available: 
\begin{itemize}
    \item BioNTech-Pfizer - (2020/12/21)
    \item Moderna - (2021/01/06)
    \item AstraZeneca - (2021/01/29)
    \item Johnson \& Johnson - (2021/03/11)
\end{itemize}
But it's not the vaccines that will stop the pandemic, it's the vaccination. 
\\





\section{Portugal}

\subsection{Brief description}
Portugal is a country in Europe with a resident population of around 10.2 million inhabitants.  \\
According to the local statistics institute, INE (Instituto Nacional de Estatística - National Statistics Institute), in the publication dated 2019 and edited in 2020, called \cite{pubINE}"Demographic Statistics - 2019", we can find some interesting information such as:
\begin{itemize}
    \item In percentage terms, relative to \textbf{sex},
    \begin{itemize}
        \item \textbf{47,2\textdiscount} of the population is \textbf{male};
        \item \textbf{52,8 \textdiscount} of the population is \textbf{female}.
    \end{itemize}  
    \item In terms of \textbf{age} percentages, 
    \begin{itemize}
        \item \textbf{13,6\textdiscount} of \textbf{young people} (0-14 y/o);
        \item \textbf{64.3\textdiscount} of people of \textbf{working age} (15-64 y/o);
        \item \textbf{22.1\textdiscount} of \textbf{elderly people} (65+ y/o). 
    \end{itemize}
\end{itemize}

Se calhar por aqui alguns graficos relativos ao descrito acima nao é má ideia.

\subsection{Regions}
In terms of regions, Portugal is usually divided in seven groups.


\begin{figure}[h]
\centering % para centralizarmos a figura
\includegraphics[width=10cm]{images/portugal.png} 

\caption{Portugal's Regions}
\label{figura:qualquernome}
\end{figure}

We'll describing them below, also mentioning some statistical data that may be interesting. \\
All information related to the labor market in the regions was obtained by consulting the \cite{trab} website of the European Commission. The demographic percentages presented are based on data from the National Institute of Statistics of Portugal. 

\subsubsection{North}
The employment structure in the North region presents 3 sub-regions with specific characteristics: 
The Porto Metropolitan Area, with a strong incidence of services (mainly trade) with greater technological and knowledge intensity. A surrounding shore, where industrial employment is  higher than the national average and rural areas, where nearly half of employment is concentrated in agriculture or non-commercial services.

In terms of demographics percentages:
    \begin{itemize}
        \item Population density: {34,73\textdiscount} . 
        \item Sex percentages: {47,2\textdiscount} male and {52,8\textdiscount} female.
        \item Age percentages: 
        \begin{itemize}
        \item {12,63 \textdiscount} aged 14 and younger;
        \item {66,43\textdiscount} aged 15 to 64;
        \item {20,94\textdiscount} aged 65 and older.
        \end{itemize}
    \end{itemize}

\subsubsection{Center}
In this region, the Services sector is the most relevant in terms of employment - with emphasis on Trade and Vehicle Repair, Health and Social Support Services and the Education. \\
In terms of demographics percentages:
    \begin{itemize}
        \item Population density: {21,54\textdiscount} . 
        \item Sex percentages: {47,42\textdiscount} male and {52,58\textdiscount} female.
        \item Age percentages: 
        \begin{itemize}
        \item {12,05 \textdiscount} aged 14 and younger;
        \item {63,42\textdiscount} aged 15 to 64;
        \item {24,53\textdiscount} aged 65 and older.
        \end{itemize}
    \end{itemize}
    
\subsubsection{Metropolitan area of Lisbon }
This is the region with the highest population density in the country. It's the region with the highest concentration of services, with emphasis on services provided mostly by the Public Sector; Education; Health and social support services.\\
In terms of demographics percentages:
    \begin{itemize}
        \item Population density: {27,81\textdiscount} . 
        \item Sex percentages: {46,71\textdiscount} male and {53,29\textdiscount} female.
        \item Age percentages: 
        \begin{itemize}
        \item {15,88 \textdiscount} aged 14 and younger;
        \item {62,04\textdiscount} aged 15 to 64;
        \item {22,08\textdiscount} aged 65 and older.
        \end{itemize}
    \end{itemize}
    
\subsubsection{Alentejo}
This is the region with the lowest population density in the country. Most of the region's territory is dedicated to Agriculture, allied to cattle breeding and also forestry.\\
In terms of demographics percentages:

    \begin{itemize}
        \item Population density: {6,84\textdiscount} . 
        \item Sex percentages: {47,97\textdiscount} male and {52,03\textdiscount} female.
        \item Age percentages: 
        \begin{itemize}
        \item {12,4 \textdiscount} aged 14 and younger;
        \item {62,05\textdiscount} aged 15 to 64;
        \item {25,55\textdiscount} aged 65 and older.
        \end{itemize}
    \end{itemize}
    
\subsubsection{Algarve}
The economic structure of this region is based on 5 strategic sectors associated with the region's natural resources: hospitality, catering and tourism, health, creative activities, agri-food and maritime activities. 
In terms of demographics percentages:
    \begin{itemize}
        \item Population density: {34,73\textdiscount} . 
        \item Sex percentages: {47,2\textdiscount} male and {52,8\textdiscount} female.
        \item Age percentages: 
        \begin{itemize}
        \item {12,63 \textdiscount} aged 14 and younger;
        \item {66,43\textdiscount} aged 15 to 64;
        \item {20,94\textdiscount} aged 65 and older.
        \end{itemize}
    \end{itemize}
    
\subsubsection{Azores autonomous region}
The region's economy is fundamentally based on activities mainly from the Public Sector (Public Administration, Social Security, Education, Health and Social Support activities). The activities of Trade and Repair of Vehicles and Accommodation and Restoration are equally important in employment in the region. \\
In terms of demographics percentages:
    \begin{itemize}
        \item Population density: {2,36\textdiscount} . 
        \item Sex percentages: {48,55\textdiscount} male and {51,45\textdiscount} female.
        \item Age percentages: 
        \begin{itemize}
        \item {15,37 \textdiscount} aged 14 and younger;
        \item {69,69\textdiscount} aged 15 to 64;
        \item {14,99\textdiscount} aged 65 and older.
        \end{itemize}
    \end{itemize}
    
\subsubsection{Madeira autonomous region }
In terms of more established work areas, this region is very similar to the Azores region. Tourism has an important role for both regions.\\
In terms of demographics percentages:
    \begin{itemize}
        \item Population density: {2,47\textdiscount} . 
        \item Sex percentages: {46,67\textdiscount} male and {53,33\textdiscount} female.
        \item Age percentages: 
        \begin{itemize}
        \item {13,11 \textdiscount} aged 14 and younger;
        \item {69,91\textdiscount} aged 15 to 64;
        \item {16,98\textdiscount} aged 65 and older.
        \end{itemize}
    \end{itemize}

\\
\\
\\
Possivelmente alguns graficos aqui também seriam interessantes \\ 



\section{SARS-CoV-2 in Portugal}
\subsection{Brief description}

SARS-CoV-2 (Severe Acute Respiratory Syndrome coronavirus) is a new type of coronavirus that causes a respiratory disease called coronavirus disease 19, know as COVID-19. It was first detected in December 2019 has quickly spread globally.

Portugal recorded the first confirmed case of COVID-19 on March 2 and the first death occurred
on March 16. Since then we have added more than 800 thousand registered cases and more than 16 thousand deaths.

Portugal was severely affected by COVID-19, being considered, in the middle of January 2021, as the worst country in the world in terms of infection and mortality rates per million inhabitants and the worst country in Europe with the highest average of cases COVID-19 daily reports, with Portugal reaching a maximum of 16432 cases and 303 deaths on 28 January 2020.












\subsection{Vaccination}

   To end this pandemic, a large part of the world needed to be immune to the virus. The safest way to achieve this was through a vaccine, so with the help of investments by companies, governments, international health organizations and university research groups it was possible to develop a series of vaccines including Phyzer, Moderna, Astrazeneca, Johnson & Johnson, among others.

   The first vaccine administered in Portugal was on December 27, 2020. The first vaccine was António Sarmento, 65, director of the Infectious Diseases Service, at Hospital de São João, in Porto and since then more than 2 million doses of vaccine have been administered. Despite the fact that the entire Portuguese population has access to a vaccine, depending on their clear clinical condition, Portugal opted for a vaccination plan divided into 3 phases, that is, priority groups were defined, as they are more vulnerable to COVID-19, as an example health professionals, professionals and residents of residential structures for the elderly and similar institutions, who were part of the first vaccination phase.


\chapter{Application}
\section{Research}
\subsection{Data}
\section{Implementation}
\subsection{Depois ver mais cenas}
\section{Results}
\subsection{Publication}
\subsection{Versatility}

\chapter{Conclusion}


\bibliographystyle{plain}
\begin{thebibliography}{10}

\bibitem{R-project website} \textit{\href{https://www.r-project.org/about.html}{https://www.r-project.org/about.html}} , \textbf{About/R-Project website}

\bibitem{shiny}\textit{\href{https://shiny.rstudio.com/l}{https://shiny.rstudio.com}} , \textbf{Shiny website
}
\bibitem{SARS-CoV-2}\textit{\href{https://apps.who.int/iris/bitstream/handle/10665/332197/WHO-2019-nCoV-FAQ-Virus_origin-2020.1-eng.pdf}{https://apps.who.int/.../origin-2020.1-eng.pdf}} ,\textbf{ Origin of SARS-CoV-2 - \ac{WHO}}

\bibitem{EU}\textit{\href{https://ec.europa.eu/info/live-work-travel-eu/coronavirus-response/public-health/eu-vaccines-strategy_pt}{https://ec.europa.eu/info/live-work-travel-eu/coronavirus-response/public-health/eu-vaccines-strategy_pt}} , \textbf{EU vaccine strategy} 

\bibitem{reuters}\textit{\href{https://graphics.reuters.com/world-coronavirus-tracker-and-maps/vaccination-rollout-and-access/}{https://graphics.reuters.com/world-coronavirus-tracker-and-maps/vaccination-rollout-and-access/}}, \textbf{Covid-19 vaccination tracker, Reuters} 


\bibitem{pubINE}\textit{\href{https://www.ine.pt/xurl/pub/71882686}{https://www.ine.pt/xurl/pub/71882686}}, \textbf{Instituto Nacional de Estatística - Estatísticas Demográficas : 2019. Lisboa : INE, 2020.} 


\bibitem{trab}\textit{\href{https://ec.europa.eu/eures/main.jsp?acro=lmi&lang=pt&countryId=PT&catId=57&parentId=0}{https://ec.europa.eu/eures/main.jsp?acro=lmi&lang=pt&countryId=PT&catId=57&parentId=0}}, \textbf{Labor Market Information - European Commission} 

\end{thebibliography}














\end{document}

