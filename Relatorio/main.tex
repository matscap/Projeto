\documentclass[11pt,a4paper]{report}

\usepackage[english]{babel} 
\usepackage[utf8]{inputenc} % Unicode
\usepackage{graphicx} 
\usepackage{url} 
\usepackage{textcomp} 
\usepackage{multirow} 
\usepackage[pdftex,hidelinks]{hyperref}
\usepackage{listings}
\usepackage{natbib}
\usepackage{graphicx}
\lstset{language=Java,breaklines=true,basicstyle=\scriptsize\ttfamily,frame=single,showstringspaces=false}

\title{Development of a Shiny application to visualize SARS-CoV-2 vaccination data of Portugal}


\author{
  Bruna Araújo\\
  \texttt{A84408}
  \and
  Matias Capitão\\
  \texttt{A82726} \\ 
  \\ Mentoring:\\
  Stora Cecília \\
  \and
  Rafael Antunes\\
  \texttt{A77457}
}
\date{May or June 2021}



\begin{document}

\maketitle

\begin{abstract}
% Apagar daqui
    ************* Ao fim retornamos aqui.\\ Entretanto encontrei isto para nos ajudar a fazer o Abstract *************\\ \\
%até aqui
    Write the abstract at the very end, when you’ve completed the rest of the text. There are four things you need to include:\\ \\

    Your research problem and objectives\\
    Your methods\\
    Your key results or arguments\\
    Your conclusion

\end{abstract}

\tableofcontents
\newpage

\listoffigures
\newpage

\listoftables
\newpage


\chapter{Introduction}
\section{R and Shiny}
% Apagar daqui
    ************* Poderiamos fazer uma descrição brévia (ou detalhada se quiserem) de o que é o R mas mais propriamente o que é o R+Shiny *************\\ \\
%até aqui

\section{SARS-CoV-2}
\subsection{Brief description}
% Apagar daqui
    ************* Poderiamos fazer uma descrição brévia do Covid-19. Tipo o que é e como é transmitido. Na minha opinião acho que não é preciso alongar muito esta parte dado que a nossa cena é a vacinação.   *************\\ \\
%até aqui
\subsection{Vaccination}
% Apagar daqui
    ************* Descrição da vacinação   *************\\ \\
%até aqui


\section{Portugal}
\subsection{Brief description}

\subsection{Regions}
% Apagar daqui
    ************* Se calhar uma breve descrição das regiões. Pode ser interessante em termos estatisticos. Cenas tipo, densidade demográfica, média de idades, tipos de trabalhos etc...   *************\\ \\
%até aqui

\section{SARS-CoV-2 in Portugal}
\subsection{Brief description}
% Apagar daqui
    ************* Poderiamos fazer uma descrição brévia do Covid-19 em Portugal. a primeira vez que foi detetado, zonas,regiões) de maior contágio, alturas (datas) de maior contágio...   *************\\ \\
%até aqui
\subsection{Vaccination}
% Apagar daqui
    ************* Descrição da vacinação em Portugal. A mesma cena de a pouco  *************\\ \\
%até aqui

\chapter{Application}
\section{Research}
\subsection{Data}
\section{Implementation}
\subsection{Depois ver mais cenas}
\section{Results}
\subsection{Publication}
\subsection{Versatility}

\chapter{Conclusion}







\bibliographystyle{plain}
\bibliography{references}
\end{document}
